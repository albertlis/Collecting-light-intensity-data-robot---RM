% !TeX encoding = UTF-8
% !TeX spellcheck = pl_PL

% $Id:$

%Author: Wojciech Domski
%Szablon do ząłożeń projektowych, raportu i dokumentacji z steorwników robotów
%Wersja v.1.0.0
%


%% Konfiguracja:
\newcommand{\kurs}{Sterowniki robot\'{o}w}
\newcommand{\formakursu}{Projekt}

%odkomentuj właściwy typ projektu, a pozostałe zostaw zakomentowane
%\newcommand{\doctype}{Za\l{}o\.{z}enia projektowe} %etap I
\newcommand{\doctype}{Raport} %etap II
%\newcommand{\doctype}{Dokumentacja} %etap III

%wpisz nazwę projektu
\newcommand{\projectname}{Humanistycznie upo\'sledzony robot akrobatyczny}

%wpisz akronim projektu
\newcommand{\acronim}{HURA}

%wpisz Imię i nazwisko oraz numer albumu
\newcommand{\osobaA}{Albert \textsc{Lis}, 235534}
%w przypadku projektu jednoosobowego usuń zawartość nowej komendy
\newcommand{\osobaB}{Micha\l{} \textsc{Moru\'n}, 235986}

%wpisz termin w formie, jak poniżej dzień, parzystość, godzina
\newcommand{\termin}{sr TP15 }

%wpisz imię i nazwisko prowadzącego
\newcommand{\prowadzacy}{mgr in\.{z}. Wojciech \textsc{Domski}}

\documentclass[10pt, a4paper]{article}

\include{preambula}
	
\begin{document}

\def\tablename{Tabela}	%zmienienie nazwy tabel z Tablica na Tabela

\begin{titlepage}
	\begin{center}
		\textsc{\LARGE \formakursu}\\[1cm]		
		\textsc{\Large \kurs}\\[0.5cm]		
		\rule{\textwidth}{0.08cm}\\[0.4cm]
		{\huge \bfseries \doctype}\\[1cm]
		{\huge \bfseries \projectname}\\[0.5cm]
		{\huge \bfseries \acronim}\\[0.4cm]
		\rule{\textwidth}{0.08cm}\\[1cm]
		
		\begin{flushright} \large
		\emph{Skład grupy:}\\
		\osobaA\\
		\osobaB\\[0.4cm]
		
		\emph{Termin: }\termin\\[0.4cm]

		\emph{Prowadzący:} \\
		\prowadzacy \\
		
		\end{flushright}
		
		\vfill
		
		{\large \today}
	\end{center}	
\end{titlepage}

\newpage
\tableofcontents
\newpage

%Obecne we wszystkich dokumentach
\section{Konfiguracja mikrokontrolera}
\subsection{Konfiguracja pinów}

\begin{table}[H]
	\centering
	\begin{tabular}{|l|l|l|l|}
		\hline
		PIN & Tryb pracy & Funkcja/etykieta\\
		\hline
		A0& Analog Input & LEFT\_DOWN\_PIN \\
		A1& Analog Input & RIGHT\_DOWN\_PIN \\
		A2& Analog Input & LEFT\_UP\_PIN \\
		A3& Analog Input & RIGHT\_UP\_PIN \\
		0& Rx & RX \\
		1& Tx & TX \\
		5& Digital Output & SERVO\_PIN \\
		6& Digital Output & DIRECTION \\
		7& Digital Output & STP \\
		\hline
	\end{tabular}
	\caption{Konfiguracja pinów mikrokontrolera}
\end{table}

%Obecne we wszystkich dokumentach
%Obecne w dokumencie do etapu II oraz III
\section{Urządzenia zewnętrzne}
%\textcolor{red}{Rozdział ten powinien zawierać opis i konfigurację %wykorzystanych ukladów
%	zewnętrznych, jak np. akcelerometr.}


%Obecne w dokumencie do etapu II oraz III
\section{Projekt elektroniki}
	\subsection{Schemat elektryczny}
	\begin{figure}[H]
		\centering
		\includegraphics[width=1\textwidth]{figures/schemeit-project.png}
		\caption{Schemat elektryczny}
		\label{fig:Schemat elektryczny}
	\end{figure}
	
\subsection{Regulacja położenia wertykalnego}
Do regulacji położenia wertykalnego robota wykorzystano serwo zamontowane na platformie. Jest ono sterowane za pomocą sygnału PWM z Arduino. Obok silnika krokowego zapewnia poruszanie się robota w kierunku światła. Serwo odpowiada za dokładne ustawienie prostopadle do słońca, podwyższając lub obniżając ścianki na których są zamontowane fotorezystory, natomiast silnik krokowy obraca postawę platformy w kierunku światła.

\begin{figure}[H]
	\centering
	\includegraphics[width=0.8\textwidth]{figures/pwm.png}
	\caption{Schemat poglądowy regulacji prędkości obrotowej silnika}
	\label{fig:KonfiguracjaPWM}
\end{figure}

%Obecne w dokumencie do etapu II oraz III
\section{Konstrukcja mechaniczna}

	\begin{figure}[H]
		\centering
		\includegraphics[width=1\textwidth]{figures/20190410_135905.jpg}
		\caption{Zdjęcie części mechanicznej nr 1}
		\label{fig:Zdjęcie części mechanicznej nr 1}
	\end{figure}
	
		\begin{figure}[H]
		\centering
		\includegraphics[width=1\textwidth]{figures/20190410_135853.jpg}
		\caption{Zdjęcie części mechanicznej nr 2}
		\label{fig:Zdjęcie części mechanicznej nr 2}
	\end{figure}

%Obecne w dokumencie do etapu II oraz III
\section{Opis działania programu}

\subsection{Schemat działania programu}
	\begin{figure}[H]
		\centering
		\includegraphics[width=0.8\textwidth]{figures/diagramPWM.png}
		\caption{Schemat działania programu}
		\label{fig:diagramPWM}
	\end{figure}

\subsection{Funkcja czytająca natężenie światła}
Jest odpowiedzialna za odczyt wartości i umieszczenie ich w tablicy.
	\begin{lstlisting}[tabsize=2]
	void ReadLight()
	{
		values[0] = analogRead(LEFT_DOWN_PIN);
		values[1] = analogRead(RIGHT_DOWN_PIN);
		values[2] = analogRead(LEFT_UP_PIN);
		values[3] = analogRead(RIGHT_UP_PIN);
	}
	\end{lstlisting}

\subsection{Funkcja sterująca silnikiem krokowym}
Odpowiada za ruch platformy lewo-prawo. Realizuje poruszanie się w kierunku najintensywniejszego odczytu natężenia światła.
\begin{lstlisting}[tabsize=2]
void SetStepperPosition()
{
	//Jeśli różnica przekracza tolerancję
	if (HorizontalDiff()) {
		//Jeśli platforma jest obrócona wertykalnie w drugą stronę zmienia kierunek lewo/prawo
		if (sposition > 90) {
			//jeśli maksymalny odczyt z lewej strony
			if ((pos == 0) || (pos == 2))
			{
			digitalWrite(DIRECTION, HIGH);
			++StepCounter;
		}
		else {
			digitalWrite(DIRECTION, LOW);
			--StepCounter;
		}
	}
	else {
		if ((pos == 1) || (pos == 3)) {
			digitalWrite(DIRECTION, HIGH);
			++StepCounter;
		}
		else {
			digitalWrite(DIRECTION, LOW);
			--StepCounter;
		}
	}
	//Wykonaj krok
	digitalWrite(STP, state);
	state = !state;
	}
}
\end{lstlisting}

\subsection{Funkcja sterująca serwomechanizmem platformy}
\begin{lstlisting}[tabsize=2]
void SetServoPosition()
{
	//Jeśli przekracza tolerancję
	if (VerticalDiff()) {
	//Jeśli maksimalny odczyt na dole
	if ((pos == 0) || (pos == 1)) {
		if (sposition > 0)
			serwo.write(--sposition);
		}
		else {
			if (sposition < 180)
				serwo.write(++sposition);
		}
	}
}
\end{lstlisting}

\subsection{Funkcje odpowiadające za sprawdzenie czy różnica odczytów jest większa od tolerancji}
\begin{lstlisting}[tabsize=2]
inline bool VerticalDiff()
{
	upMax = ( values[2] > values[3] ? values[2] : values[3] );
	downMax = ( values[0] > values[1] ? values[0] : values[1] );
	return ( abs(upMax - downMax) > TOLERANCE ? true : false );
}

inline bool HorizontalDiff()
{
	leftMax = ( values[0] > values[2] ? values[0] : values[2] );
	rightMax = ( values[1] > values[3] ? values[1] : values[3]);
	return ( abs(leftMax - rightMax) > TOLERANCE ? true : false);
}
\end{lstlisting}

%Obecne w dokumencie do etapu II oraz III (jeśli coś zostało niezrealizowane)
\section{Zrealizowane zadania}
\subsection{Wizualizacja}
Udało się zrealizować wizualizację ukazującą natężenie światła w określonym miejscu w przestrzeni zrealizowaną dla przykładowych danych. Wizualizacja zakłada, że robot będzie poruszał się po polu prostokąta raz przy razie zbierając w ramach możliwości jak najdokładniejsze dane.
\begin{figure}[H]
\centering
\includegraphics[width=1\textwidth]{figures/wiz.png}
\caption{Przykładowa wizualizacja}
\end{figure}

\section{Urządzenia zewnętrzne}
Wykorzystywanym w projekcie urządzeniem zewnętrznym jest czujnik natężenia światła -- GY--30--BH1750. Czujnik został zamontowany na górze platformy z fotorezystorami w celu zbierania informacji potrzebnych do wizualizacji.
\begin{figure}[H]
\centering
\includegraphics[width=6cm]{figures/gy.png}
\caption{Czujnik wykorzystany na platformie}
\end{figure}


\section{Podsumowanie}

Udało się zrealizować większość zadań. Nastąpiły drobne zmiany koncepcyjne jak użycie potencjometru do regulacji prędkości obrotowej napędu. To będzie wymagać mniejszej ingerencji gdy będziemy projektować regulator PID.

\newpage
\addcontentsline{toc}{section}{Bibilografia}
\bibliography{bibliografia}
\bibliographystyle{plabbrv}


\end{document}







































