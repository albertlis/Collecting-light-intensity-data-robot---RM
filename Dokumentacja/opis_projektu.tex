% !TeX encoding = UTF-8
% !TeX spellcheck = pl_PL

% $Id:$

%Author: Wojciech Domski
%Szablon do ząłożeń projektowych, raportu i dokumentacji z steorwników robotów
%Wersja v.1.0.0
%


%% Konfiguracja:
\newcommand{\kurs}{Roboty Mobilne}
\newcommand{\formakursu}{Projekt}

%odkomentuj właściwy typ projektu, a pozostałe zostaw zakomentowane
\newcommand{\doctype}{Za\l{}o\.{z}enia projektowe} %etap I
%\newcommand{\doctype}{Raport} %etap II
%\newcommand{\doctype}{Dokumentacja} %etap III

%wpisz nazwę projektu
\newcommand{\projectname}{Robot mobilny z samopozycjonującą się platformą fotowoltaniczną}

%wpisz akronim projektu
%\newcommand{\acronim}{WK}

%wpisz Imię i nazwisko oraz numer albumu
\newcommand{\osobaA}{Paula \textsc{Langkafel}, 235373}
%w przypadku projektu jednoosobowego usuń zawartość nowej komendy
\newcommand{\osobaB}{Albert \textsc{Lis}, 235534}

\newcommand{\osobaC}{Michał \textsc{Moruń}, 235986}

%wpisz termin w formie, jak poniżej dzień, parzystość, godzina
\newcommand{\termin}{wtorek TP 17}

%wpisz imię i nazwisko prowadzącego
\newcommand{\prowadzacy}{mgr in\.{z}. Michał \textsc{Błędowski}}

\documentclass[10pt, a4paper]{article}

\include{preambula}
	
\begin{document}

\def\tablename{Tabela}	%zmienienie nazwy tabel z Tablica na Tabela

\begin{titlepage}
	\begin{center}
		\textsc{\LARGE \formakursu}\\[1cm]		
		\textsc{\Large \kurs}\\[0.5cm]		
		\rule{\textwidth}{0.08cm}\\[0.4cm]
		{\huge \bfseries \doctype}\\[1cm]
		{\huge \bfseries \projectname}\\[0.5cm]
%		{\huge \bfseries \acronim}\\[0.4cm]
		\rule{\textwidth}{0.08cm}\\[1cm]
		
		\begin{flushright} \large
		\emph{Skład grupy:}\\
		\osobaA\\
		\osobaB\\
		\osobaC\\[0.4cm]
		
		\emph{Termin: }\termin\\[0.4cm]

		\emph{Prowadzący:} \\
		\prowadzacy \\
		
		\end{flushright}
		
		\vfill
		
		{\large \today}
	\end{center}	
\end{titlepage}

\newpage
\tableofcontents
\newpage

%Obecne we wszystkich dokumentach
\section{Opis projektu}
Celem projektu jest stworzenie robota sterowanego za pomocą akcelerometru w telefonie. Dane będą przesyłanie za pomocą Wi-Fi. Robot będzie zasilany ogniwami fotowoltanicznymi. W celu maksymalizacji uzyskanej energii słonecznej platforma powinna pozycjonować się prostopadle do padającego światła. Sterowanie robotem jest odseparowane od sterowania platformą.

	\begin{figure}[H]
		\centering
		\includegraphics[width=0.8\textwidth]{diag1.jpg}
		\caption{Architektura robota}
		\label{fig:Architektura robota}
	\end{figure}

	\begin{figure}[H]
		\centering
		\includegraphics[width=0.8\textwidth]{diag2.jpg}
		\caption{Architektura platformy}
		\label{fig:Architektura platformy}
	\end{figure}
\section{Założenia projektowe}

\subsection{Mechanika}
\begin{enumerate}
	
	\item Sterowanie platformą
	\newline
	Realizowane w oparciu o dwa serwomechanizmy. Jeden będzie odpowiedzialny za pozycjonowanie wertykalne (serwo $180^\circ$) natomiast drugi za horyzontalne (serwo $360^\circ$).
	
	\item Podstawa oraz separator fotorezystorów
	\newline
	Zbudowana z klocków lego. Posiada duże możliwości dopasowania do zmian w trakcie projektu.
	
\end{enumerate}

\subsection{Elektronika}
\begin{enumerate}
	\item Mikrokontrolery
	\newline
	Do sterowania robotem zostanie użyty sterownik STM32L476GDiscovery. Natomiast do realizacji pozycjonowania platformy zostanie użyty sterownik Arduino Mega2560.
	
	\item Zasilanie
	\newline
	Oparte o akumulatory li-ion 18650 lub powerbank. Ustalenie napięcia 5V za pomocą przetwornicy step-up MT3608 do zasilania płytki Arduino oraz serwomechanizmów. Dodatkowo użycie przetwornicy step-down do napięcia 3.3V w celu zasilenia mikrokontrolera STM32L476GDiscovery i modułu Wi-Fi.
	
	\item Czujniki
	\newline
	Użyte zostaną 4 odseparowane fotorezystory gl5516. Ich wartości mierzone będą za pomocą portów analogowych  sterownika Arduino Mega2560.
\end{enumerate}

\section{Harmonogram pracy}

\subsection{Zakres prac}
\begin{enumerate}
	\item Zapoznanie się z mikrokontrolerem
	\newline
	Wykorzystane to tego celu zostaną poradniki ze strony www.forbot.pl.
\end{enumerate}
\subsection{Kamienie milowe}
\begin{enumerate}
	\item Zbudowanie robota
	\item Implementacja modułu elektronicznego do platformy
	\item Implementacja algorytmów sterowania platformą
\end{enumerate}

\subsection{Diagram Gantta}

	\begin{figure}[H]
		\centering
		\includegraphics[width=1.1\textwidth]{gantt.png}
		\caption{Diagram Gantta}
		\label{fig:DiagramGantta}
	\end{figure}
\subsection{Podział prac}
\begin{table}[H]
	\centering
	\begin{tabular}{|c|c|c|}
		\hline
		Paula Langkafel                               & Albert Lis                                             & Michał Moruń        \\ \hline
		\multicolumn{3}{|c|}{Zapoznanie się z programem CubeMX oraz jego konfiguracją}                                               \\ \hline
		\multicolumn{3}{|c|}{Zbudowanie ramy robota}                                                                                 \\ \hline
		\multicolumn{3}{|c|}{Implementacja modułu elektronicznego umożliwiającą poruszanie się robota za pomocą telefonu}            \\ \hline
		Implementacja elektroniki  & Budowanie odpowiednich & Budowanie  \\
		 sterującą platformą & algorytmów stresujące platformą &  platformy  \\ \hline
		\multicolumn{3}{|c|}{Integracja wszystkich modułów}                                                                          \\ \hline
	\end{tabular}
\end{table}

\end{document}